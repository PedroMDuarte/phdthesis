\documentclass[a4paper]{article}
\title{A model for the MOTT featured Gaussian Column Density Profile}
\usepackage{float}%to fix figure position
\usepackage{graphicx}%to include figure
% Change "article" to "report" to get rid of page number on title page
\usepackage{amsmath,amsfonts,amsthm,amssymb}
\usepackage{setspace}
\usepackage{Tabbing}
\usepackage{fancyhdr}
\usepackage{lastpage}
\usepackage{extramarks}
\usepackage{chngpage}
\usepackage{soul,color}
\usepackage{graphicx,float,wrapfig}

% In case you need to adjust margins:
\topmargin=-0.45in      %
\evensidemargin=0in     %
\oddsidemargin=0in      %
\textwidth=6.5in        %
\textheight=9.0in       %
\headsep=0.25in         %
\setlength\parindent{0pt}
\begin{document}
\begin{spacing}{1.25}
\maketitle

In order to derive the density and atom number properly when the density profile of the cloud has a "MOTT" feature, 
we need to use a fit function which is a project of a 3D Gaussian with a constant core.\\ 
Before go into the spacial "MOTT" case. Let's review the normal Gaussian distribution first.
\section{3D Gaussian Density Profile}
In a normal case, we assume the atom cloud has a density profile of Gaussian in 3D:
\begin{equation}
n(\vec{r}) = n_{0} \times e^{-(\dfrac{x^{2}}{w_{x}^{2}}+\dfrac{y^{2}}{w_{y}^{2}}+\dfrac{z^{2}}{w_{z}^{2}})}
\end{equation}
The column density of it is:
\begin{equation}
n_{col}(\vec{r}) = \int_{-\infty}^\infty n(\vec{r}) dz= w_{z}\times\sqrt{\pi}\times n_{0}\times e^{-(\dfrac{x^{2}}{w_{x}^{2}}+\dfrac{y^{2}}{w_{y}^{2}})}
\end{equation}
The total atom number is:
\begin{equation}
N_{tot} = \int_{-\infty}^\infty\int_{-\infty}^\infty n_{col}(\vec{r}) dxdy = w_{x}w_{y}w_{z}\times \pi^{\frac{3}{2}} \times n_{0}
\end{equation}
In the experiment, we will get column density of the atom from either phase contrast of absorption image.
We then fit the column density to the equation:
\begin{equation}
n_{col}(x,y) = A \times e^{-(\dfrac{x^{2}}{w_{x}^{2}}+\dfrac{y^{2}}{w_{y}^{2}})}
\end{equation}
Compare this with equation (2) we find that the peak density $n_{0}$ and total number can be derived from fitting parameter $A$,$w_{x}$,$w_{x}$:
\begin{equation}
n_{0} = \frac{A}{\sqrt{\pi}\times w_{z}} = \frac{A}{\sqrt{\pi}\times \sqrt{w_{x}w_{y}}}
\end{equation}

\begin{equation}
N_{tot} = A \times\pi\times  w_{x}\times w_{y}  
\end{equation}
Here since we don't have the information of $w_{z}$. We make an assumption that $w_{z}=\sqrt{w_{x}w_{y}}$.

\newpage
\section{3D Gaussian with A Constant Core Density Profile}
We define the density profile of this case as:

\begin{equation}
n(\vec{r}) = 
\begin{cases}
n_{0} \cdot e^{-{r_{d}}^2},  r_{d} > r_{0} \\
n_{0} \cdot e^{-{r_{0}}^2},  else\\
\end{cases}
\end{equation}
\begin{equation}
r_{d} \equiv (\dfrac{x^{2}}{w_{x}^{2}}+\dfrac{y^{2}}{w_{y}^{2}}+\dfrac{z^{2}}{w_{z}^{2}})^{\frac{1}{2}}
\end{equation}
Where $r_{0}$ is a constant. If $r_{0} =0$, the profile is a regular Gaussian.
The column density of it is:

\begin{equation}
n_{col}(x,y) = \int_{-\infty}^\infty n(\vec{r}) dz=
\begin{cases}
 w_{z}\cdot\sqrt{\pi}\cdot n_{0}\cdot e^{-{r_{xy}}^2},  r_{xy}>r_{0}\\
  w_{z}\cdot\sqrt{\pi}\cdot n_{0}\cdot e^{-{r_{xy}}^2}\cdot[1-erf({r_{xy}}^*)+\frac{2}{\sqrt{\pi}}\cdot  r_{xy}^* \cdot e^{-(r_{xy}^*)^2}],  else\\
 \end{cases}
\end{equation}
\begin{equation}
r_{xy} \equiv (\dfrac{x^{2}}{w_{x}^{2}}+\dfrac{y^{2}}{w_{y}^{2}})^{\frac{1}{2}}
\end{equation}
\begin{equation}
r_{xy}^* \equiv ({r_0}^2-{r_{xy}}^2)^{\frac{1}{2}}
\end{equation}
\begin{equation}
erf(x) = \frac{2}{\sqrt{\pi}}\cdot \int_0^x e^{-t^2} dt
\end{equation}
The peak density and numbers:

\begin{equation}
A \equiv w_{z}\cdot\sqrt{\pi}\cdot n_{0}
\end{equation}
\begin{equation}
n_{0} = \frac{A\cdot e^{-{r_{0}}^2}}{\sqrt{\pi}\times w_{z}} = \frac{A\cdot e^{-{r_{0}}^2}}{\sqrt{\pi}\times \sqrt{w_{x}w_{y}}}
\end{equation}
\begin{equation}
N_{tot} = A\cdot \pi \cdot w_{x}w_{y} \times
[1-erf(r_{0})+
\frac{2\cdot r_{0}\cdot e^{-{r_{0}}^2}}{\sqrt{\pi}}(1+
\frac{2 r_{0}^{2} }{3})]
\end{equation}
\end{spacing}
\end{document}

