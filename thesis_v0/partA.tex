%%%%%%%%%%%%%%%%%%%%%%%%%%%%%%%%%%%%%%%%%%%%%%%%%%%%%%%%%%%%%%%%%%%%%%%%%%%%%%%
%%%%%%%%%%%%%%%%%%%%%%%%%%%%%%%%%%%%%%%%%%%%%%%%%%%%%%%%%%%%%%%%%%%%%%%%%%%%%%%
%%%%  APPENDIX A
%%%%%%%%%%%%%%%%%%%%%%%%%%%%%%%%%%%%%%%%%%%%%%%%%%%%%%%%%%%%%%%%%%%%%%%%%%%%%%%
%%%%%%%%%%%%%%%%%%%%%%%%%%%%%%%%%%%%%%%%%%%%%%%%%%%%%%%%%%%%%%%%%%%%%%%%%%%%%%%

\chapter{Compensated optical lattice potential}
\label{app:lattice}

Experiments with ultracold atoms rely on the forces that can be applied on an
atom using light.   When the light is near resonant with an atomic transition
the force is dissipative, and can be used to dramatically slow down the motion
of atoms and bring their temperatures into the $\mu$K regime.  If the light is
far detuned, a conservative force refered to as the dipole force arises.  An
electric field induces an electric dipole moment on an atom, which then can
have a spatially dependent potential energy if the electric field is itself
inhomoegneous.  The dipole potential is an important part of nearly all
expriments with ultracold atoms at the present time.  The ability to shape the
transverse profile of a laser beam and to create interference patterns between
multiple beams, gives the experimenter a wide range of
possibilities~\cite{Grimm2000b}.  

As part of this work we have used a potential that we referred to as a
compensated lattice.  In this appendix we go into more detail
regarding the technical aspects of the compensated lattice potential.  


\section{Dipole potential and scattering rate}

The dipole potential\,\cite{Grimm2000b} produced by far-detuned light of
frequency $\omega_{L}$ on a two-level atom is given by
\begin{equation}
  U_{\text{dip}}(\bv{r}) = - 
  \frac{3\pi c^{2}}{ 2 \omega_{0}^{3} } \Gamma
%  \frac{I(\bv{r})}{\isat}
%  \frac{ \hbar \Gamma^{2}}{4} 
  \left( \frac{1}{\omega_{0} - \omega_{L}} + \frac{1}{\omega_{0} 
   + \omega_{L}}  \right) I(\bv{r}) 
\end{equation}
%where $\isat=5.1\text{mW}/\text{cm}^{2}$.

The associated photon scattering rate is 
\begin{equation}
  \Gamma_{\text{sc}} = 
  \frac{3\pi c^{2}}{ 2 \hbar \omega_{0}^{3} } \Gamma^{2} 
  \left( \frac{ \omega_{L} }{ \omega_{0} } \right)^{3} 
  \left( \frac{1}{\omega_{0} - \omega_{L}} + \frac{1}{\omega_{0} 
   + \omega_{L}}  \right)^{2} I(\bv{r})
\end{equation} 
  

The beams that we use in our lattice have a Gaussian cross section such
that the intensity (of a single beam) is given by
\begin{equation}
  I(\bv{r}) = \frac{2P}{\pi w^{2}}  
   \exp\left[ -2\frac{r_{\perp}^{2} }{w^{2}}  \right]
\end{equation}
where $r_{\perp}$ is the perpendicular distance from the beam axis and we
neglect the small variation of the intensity along the beam axis.

We will calculate the heating rate due to spontaneous emission at the center
of the potential, where the intensity of all the beams is the largest.   We define the constnats $u_{L}$ and $h_{L}$ according to 
\begin{equation}
\begin{split}
  k_{\text{B}}  u_{L} & =  
  \frac{3 \pi c^{2}}{ 2 \omega_{0}^{3} } \Gamma
  \left( \frac{1}{\omega_{0} - \omega_{L}} + \frac{1}{\omega_{0} 
   + \omega_{L}}  \right)  \\ 
 &  \\ 
 h_{L} & =
  \frac{3 \pi c^{2}}{  2 \hbar \omega_{0}^{3} } \Gamma^{2} 
  \left( \frac{ \omega_{L} }{ \omega_{0} } \right)^{3} 
  \left( \frac{1}{\omega_{0} - \omega_{L}} + \frac{1}{\omega_{0} 
   + \omega_{L}}  \right)^{2} 
\end{split}
\end{equation}
Which allows writing the dipole potential and the scattering rate simply as
\begin{equation}
\begin{split} 
 U_{0}/k_{\text{B}} &  = u_{L} I_{0} \\ 
 \Gamma_{\text{sc0}} &  = h_{L} I_{0}
\end{split}
\label{eq:uLhL}
\end{equation}
In our experiments we use lasers at 1064~nm and 532~nm, the values of $u_{L}$
and $h_{L}$ for this two wavelengths are given in Table~\ref{tab:uLhL} towards
the end of this document.

%\section{Heating}
%
%The energy change for an atom that starts at rest, absorbs a photon of
%momemtum $\bv{k}$ from the laser,  and then scatters a photon of momentum
%$\bv{k}'$ ($|\bv{k}'|=|\bv{k}|$) is given by 
%\begin{equation}
%  \Delta E = \frac{\hbar^{2}( \bv{k} - \bv{k}')^{2}}{2m} 
%           = \frac{\hbar^{2}(2k^{2} - 2 \bv{k}\cdot\bv{k}')}{2m} 
%\end{equation}
%Averaging over all directions of $\bv{k}'$ results in  
%\begin{equation}
%  \overline{\Delta E} = 2 \frac{\hbar^{2} k^{2} }{2m}  = 2 E_{R} 
%\end{equation}
%The heating rate is equal to the mean energy deposited per scattering event times the scattering rate:
%\begin{equation}
%   \dot{E}_{L}  = ( \Delta E )\Gamma_{\text{sc}0} = 
%   ( 2 E_{R,L} )\Gamma_{\text{sc}0} =( 2 E_{R,L}) h_{L} I_{0}
%\end{equation}

\section{Compensated lattice potential: definitions and simplified expressions}
\label{sec:complatt}

An optical lattice potential results due to the stationary interference pattern
of two or more laser beams.   The most common configuration, and the one
relevant to our work, consists on a laser beam that is retroreflected upon
itself, which produces a standing wave with nodes separated by half the
wavelength of the light.   Ideally the retroreflected path would have the same
power and beam waist (at the position of the atoms) as the input path, however
in practice this is not the case.   The light for the lattice is brought to the
apparatus using an optical fiber; the retro beam alignment process consists of
maximizing the light that goes back trough this fiber.   The alignment process
guarantees that (within $\sim$5\%) the input and retro beam waists will be the
same. In this treatment we will initially consider different input and retro
waists, but in the end we will set them to be equal.   On the other hand, the
retro reflected power can be significantly lower due to losses on the retro
path, so we do consider a different power for the input and retro beams
throughout. 

We define a factor, $R$,  which characterizes the lossses of the retro path,
such that the power of the retro beam is $P_{\text{r}} = P_{\text{i}} R$, where
r,i stand for retro and input respectively.    Below we tabulate the beam
waists and retro factors for the three axes of our simple cubic lattice
potential: 
\begin{center}
\begin{tabular}{c|c|c|c|c}
      & LATTICE& BEAM 1 & BEAM 2 & BEAM 3   \\ \hline \hline 
  $w  $ &  beam waist  & 48.3 & 45.9 & 41.3 \\
  $R $ &  retro factor & 0.93 & 0.77 & 0.68\\
\end{tabular}
\end{center}
The calibration method used to obtain these values will be presented in a later
section. 

In our setup we have the possibility of changing the polarization
of the retro beam.   We use a liquid crystal retarder, which allows us to
accurately set the polarization of the retro beam parallel or perpendicular to
that of the input beam.   In between these two values we can continuously set
the fraction of retro power that has the same polarization of the input beam,
and this allows us to smoothly change the potential from a dimple (minimize
retro power at input polarization) to a lattice (maximize retro power at input
polarization).    We will use the letter $\alpha$ to refer to the fraction of
power in the retro beam that has the same polairzation as the input beam, and
thus can interfere with it to form the lattice. 

\subsection{Electric field}

The electric field on the axis of our lattice (input propagating along $+z$)
can be determined as a sum of the input and retro electric fields.  We
independently treat the interfering ($\parallel$) and non-interfering ($\perp$)
fields.  
\begin{equation}
\begin{split}
  \sqrt{\frac{2}{\epsilon_{0}c}}  
   E_{1D\parallel}(x,y,z) & = Ae^{ik z} + Be^{-ikz + \phi_{\parallel}(\alpha)} \\
  \sqrt{\frac{2}{\epsilon_{0}c}}  
  E_{1D\perp}(x,y,z) & = Ce^{-ikz + \phi_{\perp}(\alpha)} \\
   & \\ 
 \text{where}  \ \ \ \ \ \ \ \ \ \ \ \ \ \ \ \  & \\
   A & = \sqrt{ \frac{2\pin }{\pi w_{\text{i}}^{2} } } 
         \exp\left[ -\frac{ x^{2} + y^{2} }{w_{\text{i}}^{2}}\right]\\ 
   B & = \sqrt{ \frac{2\pin  R \alpha}{\pi w_{\text{r}}^{2} } } 
         \exp\left[ -\frac{ x^{2} + y^{2} }{w_{\text{r}}^{2}}\right]\\ 
   C & = \sqrt{ \frac{2\pin  R  (1-\alpha)}{\pi w_{\text{r}}^{2} } } 
         \exp\left[ -\frac{ x^{2} + y^{2} }{w_{\text{r}}^{2}}\right]\\ 
\end{split}
\end{equation}

Notice that the field can have phase shifts that depend on $\alpha$.  We have
not characterized those phase shift for our retarder setup.  We will focus only
on situations where $\alpha=0$ or $1$,  so that we can neglect them.    


\subsection{Lattice depth}

To obtain the lattice depth we need to set $\alpha=1$ and look at the
oscillatory part of the interfering intensity
$(\epsilon_{0}c/2)|E_{1D\parallel}|^{2}$.  The lattice depth will have a
gaussian transverse profile as a function of $x,y$.  To get the lattice depth
at the axis of the beam we set $x=0,y=0$ and obtain
\begin{equation}
  (\epsilon_{0}c/2)|E_{1D\parallel}|^{2} = I_{1D\parallel}(z) =
  \frac{8}{\pi}  
   \sqrt{ \frac{ \pin^{2} R }{ \win^{2} \wret^{2}} } 
 \cos ^2\left(k z\right)
  - \frac{4}{\pi} \sqrt{ \frac{ \pin^{2}R }{ \win^{2}\wret^{2} }} 
+\frac{2 \pin }{\pi  \win^{2} }
+\frac{2 \pin R}{\pi  \wret^{2} }
\end{equation}
The lattice depth is the dipole potential produced by the oscillatory part of
$I_{1D\parallel}$
\begin{equation}
  s = 
  \frac{V_{\text{latt}}}{E_{R,\text{ir}}}  = 
  \frac{ k_{\text{B}} |u_{\text{ir}}| }{E_{R,\text{ir}}}
  \frac{8}{\pi}  
    \frac{ \pin \sqrt{ R } }{ \win \wret} 
\end{equation}

Assuming equal beam waists for input and retro this becomes
\begin{equation}
  s = 
  \frac{ k_{\text{B}} |u_{\text{ir}}| }{E_{R,\text{ir}}}
  \frac{8}{\pi}  
    \frac{ \pin \sqrt{ R } }{ w^{2}} 
\end{equation}
 

\subsection{Lattice and dimple radial frequencies}

To find the lattice and dimple radial frequencies we have to look at the full
intensity \[I_{1D}(r) = (\epsilon_{0}c/2)( |E_{1D\parallel}|^{2} + |E_{1D\perp}|^{2} ) \] and
set $z=0$ and $x^{2}+y^{2}=r^{2}$.

\subsubsection{Lattice $\alpha=1$}

\begin{equation}
I_{1D}(r) = 
\frac{4}{\pi} \frac{\pin \sqrt{R}}{ \win \wret} 
    \exp\left[-\frac{r^2}{\win^{2}} - \frac{r^{2}}{\wret^{2}} \right]
+
\frac{2 \pin }{\pi \win^{2}}
\exp\left[-2\frac{ r^2}{\win^{2}}\right]
+\frac{2 \pin R }{\pi \wret^{2}} 
  \exp\left[-2\frac{ r^2}{\wret^{2}}\right]
\end{equation}

Expanding the exponentials around $r=0$ gives 
\begin{equation}
I_{1D}(r) \approx I_{1D}(0)  - \frac{1}{2} 
\left[ 
\frac{8}{\pi} \frac{\pin \sqrt{R}}{ \win \wret}
 \left( \frac{1}{\win^{2}} +  \frac{1}{\wret^{2}} \right) 
+ 
\frac{2 \pin }{\pi \win^{2}} \frac{4}{\win^{2}} 
+ 
\frac{2 \pin R}{\pi \wret^{2}} \frac{4}{\wret^{2}} 
\right] r^{2}	
\end{equation}
Recall that $U(\bv{r}) = u_{L}k_{\text{B}} I(\bv{r})$.  The radial frequency is 
\begin{equation}
 \nu_{\text{latt}}^{2} = 
 \frac{|u_{\text{ir}}| k_{\text{B}} }{m\pi^{2}}
 \pin
\left[
\frac{2}{\pi} \frac{\sqrt{R}}{ \win \wret}
 \left( \frac{1}{\win^{2}} +  \frac{1}{\wret^{2}} \right) 
+ 
\frac{2  }{\pi \win^{4}} 
+ 
\frac{2 R}{\pi \wret^{4}}
\right]
\end{equation}
which for equal beam waists becomes 
\begin{equation}
 \nu_{\text{latt}}^{2} = 
 \frac{|u_{\text{ir}}| k_{\text{B}} }{m\pi^{2}}
 \pin
\left[
\frac{2 + 4 \sqrt{R} + 2R}{ \pi w^{4} }
\right]
\end{equation}

\subsubsection{Dimple $\alpha=0$}

\begin{equation}
 I_{1D}(r) = 
  \frac{2 \pin}{\pi \win^{2}} 
  \exp\left[ -2 \frac{r^{2}}{\win^{2}} \right] 
 +\frac{2 \pin R}{\pi \wret^{2}} 
  \exp\left[ -2 \frac{r^{2}}{\wret^{2}} \right]
\end{equation}
\begin{equation}
I_{1D}(r) \approx I_{1D}(0)  - \frac{1}{2} 
  \left[ 
  \frac{2 \pin}{\pi \win^{2}} \frac{4}{\win^{2}} 
  + 
  \frac{2 \pin R}{\pi \wret^{2}} \frac{4}{\wret^{2}} 
\right] r^{2}
\end{equation}
\begin{equation}
 \nu_{\text{dimp}}^{2} 	= \frac{ |u_{\text{ir}}| k_{\text{B}}   }{m\pi^{2}}
 \pin  \left[ 
  \frac{2}{\pi \win^{4}} 
  + 
  \frac{2 R}{\pi \wret^{4}} 
\right]
\end{equation}

For equal beam waists 
\begin{equation}
 \nu_{\text{dimp}}^{2} 	= \frac{ |u_{\text{ir}}| k_{\text{B}}   }{m\pi^{2}}
 \pin  \left[ 
  \frac{2+2R}{\pi w^{4}} 
\right]
\end{equation}


\textbf{Lithium mass}. 
For the mass of lithium which appears in the expressions for the radial
frequency we use the convenient expresssion 
\begin{equation}
  \frac{m}{ k_{\text{B}}}  = 6\,\frac{ \text{AMU} }{ k_{\text{B}}} 
    =  6 \frac{  h/k_{\text{B}} }{ 0.4 \,\mu\text{m}^{2}\,\text{MHz} }
   =  \frac{ 6  ( 48 \, \mu\text{K} / \text{MHz} ) }{ 0.4 \mu\text{m}^{2} \,\text{MHz} }  
    =  7.2\text{e-4}
    \frac { \mu\text{K} }{  \mu\text{m}^{2}\, \text{kHz}^{2} } 
\end{equation}
A numerical value for $\frac{ |u_{L}| k_{\text{B}} } { m \pi^{2}}$ is listed in
Table~\ref{tab:uLhL}.   
 

\subsection{Compensation beams}

Overlapped on each of our lattice axes we have a repulsive compensation beam at
a wavelength of 532~nm.  This is a single gaussian beam with a potential given
by  
\begin{equation}
  U_{\text{c}1D}( x,y) =   u_{\text{gr}}k_{\text{B}} \frac{2\pgr }{\pi \wgr^{2}} 
  \exp\left[ - 2 \frac{r^{2}}{ \wgr^{2}} \right] 
\end{equation}
and radial frequency
\begin{equation}
  \nu_{\text{gr}}^{2} =  
   \frac{ |u_{\text{gr}}| k_{\text{B}}   }{m\pi^{2}}
  \frac{2 \pgr}{\pi \wgr^{4}} 
\end{equation}

The beam waist for each of the three beams is shown below
\begin{center}
\begin{tabular}{c|c|c|c|c}
      &  COMPENSATION & BEAM 1 & BEAM 2 & BEAM 3   \\ \hline \hline 
  $w  $ &  beam waist  & 42.9 & 41.4 & 40.4 \\
\end{tabular}
\end{center}



%\section{Calibration of the compensated lattice} 

%This section is not written yet. 

At the moment we use the following compensations for each of the three axes:
\begin{center}
\begin{tabular}{ c|c|c|c}
   axis & 1 & 2 & 3 \\
   \hline
   $g_{0}\ (E_{R})$ & 3.65 &  3.90 & 2.9 \\
\end{tabular}
\end{center}
With these values we have found empirically that we can obtain the same
confinement frequencies in all three directions (which produces spherically
symmetric samples) and we can achieve $n=1$ at the center. 


%The numerical factors that show up are 
%\begin{equation}
%\begin{split}
%   \frac{E_{R}}{4u } & = 9.04\text{e-3} \equiv x_{1}\\  
%   \frac{m\pi^{2}}{u} & = 1.84\text{e-4} \equiv x_{2} 
%\end{split}
%\end{equation}
%
%In the experiment we measure the lattice depth, the trap frequencies, and the
%beam power, so the equations above are written as
%\begin{equation}
%\begin{split}
% \frac{\pin}{V_{0}} 
% \left[ \frac{\text{mW}}{E_{R}} \right]  
% & = x_{1} \frac{ \win \wret }{ \sqrt{r}}  \\
%  & \\ 
% \frac{\pin}{\nu_{Lr} ^{2}} 
% \left[ \frac{\text{mW}}{\text{kHz}^{2}} \right]  
%& = \frac{ x_{2} }{ 
%   \left( \dfrac{r}{\wret^{4} }
%      + \dfrac{ \sqrt{ r }}{ \win \wret^{3}} 
%      +  \dfrac{1}{\win^{4}} 
%      + \dfrac{ \sqrt{ r }}{ \wret \win^{3} }
%   \right)} \\ 
% & \\
% \frac{\pin}{ \nu_{Dr}^{2} }
% \left[ \frac{\text{mW}}{\text{kHz}^{2}} \right]  
% & = \frac{ x_{2} }{
%   \left(  
%     \dfrac{ 1 }{ \win^{4} } + \dfrac{ r }{ \wret^{4} } 
%    \right) } \\
%\end{split}
%\end{equation}

\section{Fermi temperature for a compensated dimple}

In our experiment we specify the powers of the IR beams by the lattice depth
that they would generate if $\alpha$ were equal to 1, that is 
% The calibration of the
%beams gives us the proportionality between power and lattice depth such that 
\begin{equation} 
%  \pin = s A_{\text{latt}} 
  \pin =  s  \frac{ E_{R,\text{ir}} }{ k_{\text{B}} |u_{\text{ir}}| } 
          \frac{ \pi \wir^{2} }{ 8 \sqrt{R}}
\end{equation}
%%Similarly the calibration of the beam also gives us the slope and offset that
%%relate power and the square of the radial frequency
%%\begin{equation} % \nu_{\text{dimp}}^{2} 	= \pin / F_{\text{dimp}}  +
%\Delta_{\text{dimp}} %= s A_{\text{latt}}/ F_{\text{dimp}}  +
%\Delta_{\text{dimp}} %\end{equation} %\begin{equation} % \nu_{\text{latt}}^{2}
%= \pin/ F_{\text{latt}}  + \Delta_{\text{latt}} %= s A_{\text{latt}}/
%F_{\text{latt}} + \Delta_{\text{latt}} %\end{equation}
\begin{equation}
 \nu_{\text{ir,dimp}}^{2} 	= s \frac{ E_{R,\text{ir}}  }{m\pi^{2}}
          \frac{ 1+R}{ 4 \wir^{2}\sqrt{R}} 
\end{equation}
\begin{equation}
 \nu_{\text{ir,latt}}^{2} 	= s \frac{ E_{R,\text{ir}}  }{m\pi^{2}}
          \frac{ 1+R+ \sqrt{R}}{ 2 \wir^{2}\sqrt{R}} 
\end{equation}
A value for $\frac{E_{R,\text{ir}}}{m\pi^{2}}$ can be found in
Table.~\ref{tab:uLhL}.

%For the green beams, the calibration gives us the proportionality factor
%between power and the square of the radial frequency \begin{equation}
%\nu_{\text{gr}}^{2} 	= \pgr F_{\text{gr}} \end{equation}
For the green beams we specify the depth of the potential produced by each beam
in units of the IR recoil, we call this quantity $g$. 
\begin{equation}
 \pgr = g \frac{E_{R,\text{ir}}}{k_{\text{B}} | u_{\text{gr}} | } 
  \frac{\pi \wgr^{2}} {2}
\end{equation}
\begin{equation}
  \nu_{\text{gr}}^{2} =  g  
   \frac{ E_{R,\text{ir}}    }{m\pi^{2}}
  \frac{1}{ \wgr^{2}} 
\end{equation}

Along each axis we have that the radial frequency of the compensated dimple
potential is
\begin{equation}
\begin{split}
  \nu_{\text{comp}}^{2}  & = \nu_{\text{ir}}^{2} - \nu_{\text{gr}}^{2}  \\
   & = 
  \frac{ E_{R,L} }{ m\pi^{2}} ( s\varphi_{\text{dimp}} - g\varphi_{\text{gr}} )
%    \equiv
%  \frac{ E_{R,L} }{ m\pi^{2}}  \varepsilon 
\end{split}
\end{equation}
where we have defined 
\begin{equation}
  \varphi_{\text{latt}}  = \frac{1+R+\sqrt{R}}{2\wir^{2}\sqrt{R}} 
 \ \ \ \ \ \ \ \ \ \ \ \  
  \varphi_{\text{dimp}}  = \frac{1+R}{4\wir^{2}\sqrt{R}} 
 \ \ \ \ \ \ \ \ \ \ \ \  
  \varphi_{\text{gr}}  = \frac{1}{\wgr^{2}}
\end{equation} 

For each of the three axes we have 
\begin{center}
\begin{tabular}{c|c|c|c|c}
      &  AXIS 1 & AXIS 2 & AXIS 3 & units   \vspace{0.1em} \\  
  $\varphi_{\text{dimp}}$  
   &2.14\text{e-4} & 2.39\text{e-4} &  2.99\text{e-4} 
   &  $\mu\text{m}^{-2}$ \\
  $\varphi_{\text{latt}}$  
   &6.433\text{e-4} & 7.160\text{e-4} &  8.903\text{e-4} 
   &  $\mu\text{m}^{-2}$ \\
  $\varphi_{\text{gr}} $ 
 & 5.434\text{e-4} & 5.834 \text{e-4} & 6.127 \text{e-4}
   &  $\mu\text{m}^{-2}$ \\
\end{tabular}
\end{center}

\vspace{1em} In order to produce spherically symmetric samples we set $s$ for
all three beams to the same (such that if $\alpha$ were equal to 1 the lattice
depths would be the same along all three directions)  and  we adjust the green
powers of beams 1 and 2 to match the compensated radial frequency of beam 3.
To calculate the Fermi temperature we will set the radial frequencies for beams
1 and 2 equal to that of beam 3, which is given by 
\begin{equation} 
 \nu_{\text{comp},3}^{2} =  
  \frac{ E_{R,\text{ir}} }{ m\pi^{2}}
  ( s\varphi_{\text{dimp},3} - g\varphi_{\text{gr},3}  ) 
\end{equation}


\vspace{0.5em}
When the dimple beams in all three axis are turned on at the same time, the
squares of the trap frequencies add up.   Beams 1,  2, 3 propagate along $x$,
$y$, and $z$ respectively.  And so we have 
\begin{equation} 
\begin{split} 
  \nu_{\text{comp},x}^{2} & = \nu_{\text{comp},2}^{2} + \nu_{\text{comp},3}^{2} \\ 
  \nu_{\text{comp},y}^{2} & = \nu_{\text{comp},3}^{2} + \nu_{\text{comp},1}^{2} \\ 
  \nu_{\text{comp},z}^{2} & = \nu_{\text{comp},1}^{2} + \nu_{\text{comp},2}^{2} \\ 
\end{split}
\end{equation}

The Fermi temperature for a spin mixture of $N$ total atoms is 
\begin{equation}
 k_{\text{B}} T_{F} = h (3N)^{1/3} \left[ \prod_{i} \nu_{x_{i}} \right]^{1/3}
\end{equation} 

\begin{equation}
\begin{split}
  T_{F} & = \frac{h}{k_{\text{B}}} (3N)^{1/3} 
 \left( 2\nu_{\text{comp},3 } \right) \\ 
 & = \frac{h}{k_{\text{B}}} 2(3N)^{1/3}
 \left( \frac{ E_{R,L}  }{m\pi^{2}} \right)^{1/2}
  ( s\varphi_{\text{dimp},3} - g\varphi_{\text{gr},3}  )^{1/2 }\\ 
 & = \left[ 48\text{e-3}\, \mu\text{K}\,\text{kHz}^{-1}\right]
   2 (3N)^{1/3}  
  \left[ 14.1 \, \mu\text{m}\,\text{kHz} \right]
  ( 2.99\, s - 6.13\, g )^{1/2 }
   \left[ 0.01 \, \mu\text{m}^{-1} \right] 
   \\ 
  & =  [ 13.5 \,\text{nK}  ] \times (3N)^{1/3} \sqrt{2.99\,s-6.13\,g} \, \\ 
%  & =  (3N)^{1/3} \sqrt{s} \, [ 14.7 \,\text{nK} ]
\end{split} 
\end{equation} 


\section{Heating}

%If we stick to the simple argument above we would find that the heating rate,
%which is proportional to the intensity could be different for red and blue
%detuned lattices.  In red detuned lattices the atoms sit in the antinodes of
%the standing wave where the intensity is the largest, whereas in a blue
%detuned lattice the atoms sit in the nodes where the intensity vanishes.   

A detailed treatment of heating as a diffusion of momentum has been carried
out by Gordon and Ashkin~\cite{Gordon1980} and also more recently
in~\cite{Gerbier2010, Pichler2010,Riou2012}.   These references stay within
the rotating wave approximation,  so here we have adapted their formulas to
include the counter-rotating term as well as a factor of
$(\omega_{L}/\omega_{0})^{3}$ that corrects the decay rate for the different
density of states at $\omega_{L}$.  From~\cite{Gordon1980} the rate of
momentum diffusion in the low saturation regime is given by
\begin{equation}
  D_{p} = \frac{\hbar^2 k^{2}}{2}  
   \frac{3\pi c^{2}}{2\hbar\omega_{0}^{3} }  
   \frac{\Gamma^{2}}{\Delta^{2}} I(\bv{r}) 
  \left( 1 + \frac{1}{k^{2}}
    \left| \frac{\nabla( \langle g|\bv{d}|e\rangle\cdot \bv{E}(\bv{r})  )}
                { \langle g|\bv{d}|e\rangle \cdot \bv{E}(\bv{r})} 
    \right|^{2} \right) 
\end{equation}
where $\bv{E}(\bv{r})e^{-i\omega t}$ is the complex classical field and
$I(\bv{r})$ is the intensity.  If the polarization of the field is
$\bv{\varepsilon}$ such that $\bv{E}(\bv{r}) = \bv{\varepsilon} E(\bv{r})$, we
have  
\begin{equation}
\begin{split}
  D_{p} & = 
  \frac{\hbar^2 k^{2}}{2}  
   \frac{3\pi c^{2}}{2\hbar\omega_{0}^{3} }  
   \frac{\Gamma^{2}}{\Delta^{2}} I(\bv{r}) 
  \left( 1 + \frac{1}{k^{2}}
    \left| 
   \frac{ \langle g|\bv{d}\cdot\bv{\varepsilon}|e\rangle\nabla E(\bv{r}) }
        { \langle g|\bv{d}\cdot\bv{\varepsilon}|e\rangle E(\bv{r})} 
    \right|^{2} \right)  \\
&  =  
  \frac{\hbar^2 k^{2}}{2}  
   \frac{3\pi c^{2}}{2\hbar\omega_{0}^{3} }  
   \frac{\Gamma^{2}}{\Delta^{2}} I(\bv{r}) 
  \left( 1 + \frac{1}{k^{2}}
    \left| 
   \frac{ \nabla E(\bv{r}) }
        {  E(\bv{r})} 
    \right|^{2} \right) 
\end{split}
\end{equation}
At this point we generalize the result of Gordon and Ashkin to far detuned light by doing the replacement
\begin{equation}
  \frac{1}{\Delta^{2}} \rightarrow 
  \left( \frac{ \omega_{L} }{ \omega_{0} } \right)^{3}
   \left(  \frac{1}{\omega_{0}-\omega_{L}} 
    + \frac{1}{\omega_{0} + \omega_{L} } \right)^{2}
\end{equation}
which results in 
\begin{equation}
D_{p}  =  
  \frac{\hbar^2 k^{2}}{2}  
   \frac{3\pi c^{2}}{2\hbar\omega_{0}^{3} }  
   \Gamma^{2}
  \left( \frac{ \omega_{L} }{ \omega_{0} } \right)^{3}
   \left(  \frac{1}{\omega_{0}-\omega_{L}} 
    + \frac{1}{\omega_{0} + \omega_{L} } \right)^{2}
   I(\bv{r}) 
  \left( 1 + \frac{1}{k^{2}}
    \left| 
   \frac{ \nabla E(\bv{r}) }
        {  E(\bv{r})} 
    \right|^{2} \right) 
\end{equation}
The momentum diffusion term is defined as 
\begin{equation}
  D_{p} = \frac{1}{2} \left( \langle p^{2} \rangle - 
   \langle \bv{p} \rangle \cdot \langle \bv{p} \rangle \right) 
\end{equation}
so for an atom that starts at rest the energy deposited per unit time is
$\dot{E} =  D_{p}/m$.  

For a plane wave propagating along $x$, $E(\bv{r})$ as defined above is
$E_{0}e^{ikx}$, so this gives a heating rate 
\begin{equation}
\begin{split}
\dot{E} &  =  
  E_{R,L}
   \frac{3\pi c^{2}}{2\hbar\omega_{0}^{3} }  
   \Gamma^{2}
  \left( \frac{ \omega_{L} }{ \omega_{0} } \right)^{3}
   \left(  \frac{1}{\omega_{0}-\omega_{L}} 
    + \frac{1}{\omega_{0} + \omega_{L} } \right)^{2}
   2  I_{0}  \\
  & = 
   2 E_{R,L}  \Gamma_{sc,0}  \\ 
  & = 
   2 E_{R,L} \frac{ h_{L} }{k_{\text{B}}|u_{L}|} U_{0} 
\end{split} 
\label{eq:heat-dipole}
\end{equation}

For a standing wave, such as that produced in a optical lattice $E(\bv{r}) =
E_{0} \cos(kx) $ and $I(\bv{r})=I_{\text{max}}\cos^{2}(kx)$ which results in
a heating rate 
\begin{equation}
\begin{split}
\dot{E} &  =  
  E_{R,L}
   \frac{3\pi c^{2}}{2\hbar\omega_{0}^{3} }  
   \Gamma^{2}
  \left( \frac{ \omega_{L} }{ \omega_{0} } \right)^{3}
   \left(  \frac{1}{\omega_{0}-\omega_{L}} 
    + \frac{1}{\omega_{0} + \omega_{L} } \right)^{2}
     I_{\max} \cos^{2}(kx) 
   \left( 1 + \frac{\sin^{2}(kx)}{\cos^{2}(kx)} \right) \\
  & = 
   E_{R,L}  \Gamma_{sc,\text{max}}
\end{split}
\end{equation}
We see that the heating rate is independent of $x$ and so it is the same for
a red or blue detuned lattice. 
  
 
 

\section{Heating due to optical lattice beams} 

\vspace{1em} As we saw in \S\ref{sec:complatt}, our lattice potential suffers
from losses on the retro path.  The intensity for one of the axis with the
light propagating along the $x$ direction (neglecting the transverse profile
$\exp[-2r_{\perp}^{2}/w^{2}]$) is given by 
\begin{equation}
\begin{split}
 I_{1D}(x) 
%& = 
%\frac{2 P_{\text{i}}}{\pi w^{2}} 
%  \left(2 \sqrt{R} \cos (2 k x )+R+1\right) \\
   & = 
\frac{2 P_{\text{i}}}{\pi w^{2}} 
  \left( 4\sqrt{R}\cos^{2}(kz) - 2\sqrt{R} +R+1\right)
\end{split}
\label{eq:I1D} 
\end{equation}
The corresponding electric field is 
\begin{equation}
\begin{split}
 E_{1D}(x) & \propto  \sqrt{\frac{2P_{\text{i}} }{\pi w^{2}}}
  e^{ikx} + \sqrt{ \frac{2P_{\text{i}}R }{ \pi w^{2} } } e^{-ikx}  \\
   & \propto 
    \sqrt{\frac{2P_{\text{i}} }{\pi w^{2}}} 2\sqrt{R}  \cos(kx) + 
    \sqrt{\frac{2P_{\text{i}} }{\pi w^{2}}} (1-\sqrt{R})e^{ikx}
\end{split} 
\end{equation}
such that
\begin{equation}
\begin{split} 
    \left| 
   \frac{ \nabla E(\bv{r}) }
        {  E(\bv{r})} 
    \right|^{2} &  =  \frac{  \left|   
    -2k\sqrt{R} \sin(kx) +  ik(1-\sqrt{R}) e^{ikx}   \right|^{2} }
    { \left| 2\sqrt{R} \cos(kx) + (1-\sqrt{R})e^{ikx} \right|^{2}}   \\ 
   &  = k^{2}
    \frac{   4R\sin^{2}(kx) + (1-\sqrt{R})^{2} +  
            2\sqrt{R}\sin(kx) (1-\sqrt{R})( 2\sin(kx) )  }
    {  4R\cos^{2}(kx)  +  (1-\sqrt{R})^{2} +
            2\sqrt{R}\cos(kx)(1-\sqrt{R}) ( 2\cos(kx) )  } \\
   &  = k^{2}
    \frac{   4\sqrt{R}\sin^{2}(kx) + (1-\sqrt{R})^{2} 
             }
    {  4\sqrt{R}\cos^{2}(kx)  +  - 2\sqrt{R} + R + 1   
             }
\end{split} 
\end{equation}

The spatially dependent factor that appears in Gordon and Ashkin's formula is
\begin{equation}
\begin{split}
   I(\bv{r}) 
  \left( 1 + \frac{1}{k^{2}}
    \left| 
   \frac{ \nabla E(\bv{r}) }
        {  E(\bv{r})} 
    \right|^{2} \right) & =  
   \frac{2 P_{\text{i}}}{\pi w^{2}} \left(
   4\sqrt{R}\cos^{2}(kz) - 2\sqrt{R} +R+1 
  + 
      4\sqrt{R} \sin^{2}(kx) +(1-\sqrt{R})^{2} 
  \right)  \\ 
   & =  
   \frac{2 P_{\text{i}}}{\pi w^{2}} 2 ( 1+R)
\end{split} 
\end{equation}
The heating rate for the 1D lattice is then
\begin{equation}
\begin{split}
  \dot{E}_{1D} & =  
   E_{R,L} 
   \frac{3\pi c^{2}}{2\hbar \omega_{0}^{3} } \Gamma^{2}
    \left( \frac{ \omega_{L} }{ \omega_{0}} \right)^{3} 
    \left(  \frac{1}{\omega_{L}-\omega_{0}} 
           + \frac{1}{\omega_{0} + \omega_{L}}  \right)^{2} 
   \frac{2 P_{\text{i}}}{\pi w^{2}} 2 ( 1+R) \\
    & = E_{R,L} h_{L}  
   \frac{2 P_{\text{i}}}{\pi w^{2}} 2 ( 1+R)  	 
\end{split}
\end{equation}
where $h_{L}$ is defined in Eq.~\ref{eq:uLhL}, and its value for 1064~nm and
532~nm appears in Table~\ref{tab:uLhL}. 


The lattice depth in our retroreflected setup is given by the $\cos^{2}$
modulation of $I_{1D}$.  This can be read off of Eq.~\ref{eq:I1D}:
\begin{equation}
  I_{\text{lattice}} = \frac{ 8 P_{\text{i}}}{\pi w^{2}} \sqrt{R} 
\end{equation}
Using the factor $u_{\lambda}$ defined above, this produces a  lattice depth
\begin{equation} 
  V_{\text{lattice}} =  k_{\text{B}} |u_{\lambda}|  
   \frac{ 8 P_{\text{i}}\sqrt{R}}{\pi w^{2}}
\end{equation}
The lattice depth in units of $E_{R}$ is usually represented by $s =
V_{\text{lattice}}/E_{R}$.    
\begin{equation}
  s =  
 \frac{k_{\text{B}}|u_{\lambda}|}{E_{R,\lambda}} 
  \frac{ 8 P_{\text{i}} \sqrt{R} }{\pi w^{2}}
 \label{eq:slatt} 
\end{equation}

This result can be incorporated into the heating rate obtained above
\begin{equation}
\boxed{
 \dot{E}_{1D} =  
   s  E_{R,L} 
   \frac{h_{L}  E_{R,L}}{ k_{\text{B}} |u_{\lambda}|} 
   \frac{  ( 1+R) }{2\sqrt{R}}}
\end{equation}
\begin{table}
\begin{center}
\begin{tabular}{c|c|c|c}
      &  1064~nm & 532~nm & units\\ 
    $u_{\lambda}$ &  -60.81 &  +61.99  &  
        $\mu\text{K}\dfrac{  \mu\text{m}^{2} }{\text{mW}}$ \\
    $h_{\lambda}$ & 0.0876 & 0.728 &  
        $ \text{s}^{-1} 
         \dfrac{\mu\text{m}^{2} }{\text{mW}}$ \\
    $ \dfrac{E_{R,\lambda}}{k_{\text{B}}}$ & 1.41 & 5.64 & $\mu\text{K}$ \\
   $ \dfrac{h_{\lambda} E_{R,\lambda}}{ k_{\text{B}} |u_{\lambda}|}$ 
      &    2.03e-3 & 6.62e-2  &  s$^{-1}$  \\
  $\dfrac{ |u_{\lambda}| k_{\text{B}} } { m \pi^{2}}$  &  8.55e3  & 8.72e3  &  
  $\frac{\mu\text{m}^{4}\,\text{kHz}^{2} }{ \text{mW} }$    \\ 
  $\dfrac{ E_{R,L} }{m \pi^{2}}$ & 198.4 & 793.68 & 
  $\mu\text{m}^{2}\, \text{kHz}^{2}$  
\end{tabular}
\end{center}
\caption[Helpful constants for dipole potential, trapping frequencies and
heating rates]{Helpful constants to calculate dipole potentials, trapping
frequencies  and heating rates.}
\label{tab:uLhL} 
\end{table}
Numerical values for  $\frac{h_{L}  E_{R,L}}{ k_{\text{B}} |u_{\lambda}|} $
are given in Table~\ref{tab:uLhL}.  We can also write down the
expression for it:
\begin{equation}
   \frac{h_{L}  E_{R,L}}{ k_{\text{B}} |u_{\lambda}|} = 
  E_{R,L} \frac{ \Gamma }{\hbar} 
   \left( \frac{\omega_{L}}{\omega_{0}} \right)^{3} 
  \left| \frac{1}{\omega_{0} - \omega_{L}} 
   + \frac{1}{\omega_{0} + \omega_{L}} \right| \approx 
  \frac{ E_{R,L}}{\hbar |\Delta|} \Gamma
\end{equation}   
where on the far right we have neglected the counter-rotating term and the
different density of states, as it is customarily done in the rotating wave
approximation (RWA).  Within RWA the heating rate is 
\begin{equation}
 \dot{E}_{1D}  = V_{\text{latt}}    \frac{ E_{R,L}}{\hbar |\Delta|} \Gamma
   \frac{  ( 1+R) }{2\sqrt{R}}
\end{equation}


\section{Heating due to green compensation beams}


Our green compensation beams are simply gaussian beams for which the heating
rate was derived in Eq.~\ref{eq:heat-dipole}.   Since we use this beams to
compensate a lattice formed by the 1064 nm IR beams we specify the
repulsive compensation at the center in units of the IR recoil,
$E_{R,\text{ir}}$:
\begin{equation} 
  g = \frac{U_{0,\text{g}}}{ E_{R,\text{ir}}}
\end{equation}
The heating rate due to each of the compenstion beams is then 
\begin{equation}
\boxed{
 \dot{E}_{g} = 
   2 E_{R,\text{ir}} \frac{ h_{g} E_{R,\text{g}} }{k_{\text{B}}|u_{g}|} g }
\end{equation}

\section{Total heating rate in the compensated lattice}

We can then add up the contributions of the IR and green to obtain 
\begin{equation}
\begin{split}
 \frac{\dot{E}_{1D,\text{tot}}}{ E_{R,\text{ir}} } & =
 \frac{ h_{ir} E_{R,\text{ir}} }{k_{\text{B}}|u_{ir}|} 
  \left(
  s \frac{1+R}{2\sqrt{R}}  + 
  2g
 \left. 
 \frac{ h_{g} E_{R,\text{g}} }{k_{\text{B}}|u_{g}|} \middle /
 \frac{ h_{ir} E_{R,\text{ir}} }{k_{\text{B}}|u_{ir}|} \right.
\right) \\
  & = 
 \frac{ h_{ir} E_{R,\text{ir}} }{k_{\text{B}}|u_{ir}|} 
  \left(
  s \frac{1+R}{2\sqrt{R}}  +
  \kappa g  
\right) \\
\end{split}  
\end{equation}
where we have defined $\kappa = \left.  \frac{ h_{g} E_{R,\text{g}}
}{k_{\text{B}}|u_{g}|} \middle / \frac{ h_{ir} E_{R,\text{ir}}
}{k_{\text{B}}|u_{ir}|} \right.$ ( For reference $\kappa = 65.22$).  Including
the contribution from the three orthogonal axes that form our lattice we
obtain
\begin{equation}
 \frac{ \dot{E}_{\text{tot}} }{ E_{R,\text{ir}} }  = 
 \frac{ h_{ir} E_{R,\text{ir}} }{k_{\text{B}}|u_{ir}|}
  \sum_{i=1,2,3} 
  \left(
  s_{i} \frac{1+R_{i}}{2\sqrt{R_{i}}}  + 
  \kappa g_{i} \right)
\end{equation} 

If we assume that the energy increase due to heating is redistributed equally
in all three dimensions we have for the rate of increase in temperature 
\begin{equation}
\boxed{ 
  \dot{T} = \frac{\dot{E}_{\text{tot}}}{ 3k_{\text{B}}} = 
  \frac{E_{R,\text{ir}}}{ 3 k_{\text{B}} }  
 \frac{ h_{ir} E_{R,\text{ir}} }{k_{\text{B}}|u_{ir}|}
  \sum_{i=1,2,3} 
  \left(
  s_{i} \frac{1+R_{i}}{2\sqrt{R_{i}}}  + 
  \kappa g_{i} \right) }
\end{equation}
Since $\overline{R}=0.8$ and  we typically use the same depth for all ir
beams and nearly the same depth for all green beams this can be approximated
by 
\begin{equation}
\begin{split}
  \dot{T} & \approx  \frac{E_{R,\text{ir}}}{  k_{\text{B}} }  
 \frac{ h_{ir} E_{R,\text{ir}} }{k_{\text{B}}|u_{ir}|}
  \left(
  s  + 
  \kappa g\right) \\
   & \approx  2.9 ( s + 65 g) \,\text{nK/s} 
\end{split}
\label{eq:numHeat}
\end{equation}
where, once again, $s$ is the lattice depth in IR recoils, and $g$ is the
compensation, also in IR recoils. It becomes obvious that our heating rate is
completely dominated by the compensating beams.   Typical values are $s=7.0$
and $g=2.9$.

Note that the large difference in the heating rate between green and IR can be
mostly atrributed to the density of states factor and the recoil energy.   The
density of states factor is $(\omega_{L}/\omega_{0})^{3} =
(\lambda_{0}/\lambda)^{3}$, which is $\sim$2 for green and $\sim$0.25  for IR,
giving a factor of 8.   The recoil scales as $\lambda^{-2}$ which gives another
factor of 4.   This accounts for a factor of 32,  the remaining factor of 2 is
due to the different heating rates for a traveling wave and a standing wave.  

\section{Total heating rate in the compensated lattice}

In dimple configuration ($\alpha=1$), and using Eq.~\ref{eq:slatt} to relate $s$ and the input power $\pin$, the 1D heating rate is 
\begin{equation}
  \dot{E} = E_{R,\text{ir}} = 
  \frac{ h_{\text{ir}} E_{R,\text{ir}} }{ k_{\text{B}} u_{\text{ir}} } 
  \frac{s}{\sqrt{R}} 
\end{equation}

Note that \[  \frac{1}{\sqrt{R}} \approx  \frac{1+R}{2\sqrt{R}} \] for the
values of $R$ that we deal with.   This means that Eq.~\ref{eq:numHeat} is
still valid for the heating rate in dimple configuration. 
